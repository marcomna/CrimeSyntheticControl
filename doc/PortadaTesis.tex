
\begin{titlepage}
{\fontfamily{phv}\selectfont
	\begin{center}
		
		\includegraphics[width=20cm]{Imágenes/Portada.png}\\[0.1in]
		
		\renewcommand{\baselinestretch}{1.5}
		\Large \textbf {MAESTRÍA EN ECONOMÍA}\\[0.025in]
		\large{TRABAJO DE INVESTIGACIÓN PARA OBTENER EL GRADO DE \\ MAESTRO EN ECONOMÍA}\\[0.3	in]
		
		\normalsize \textbf{ EL EFECTO DEL PROGRAMA \\ ``SÍ AL DESARME, SÍ A LA PAZ'' \\ SOBRE LOS DELITOS CON ARMAS DE FUEGO \\ EN LA CIUDAD DE MÉXICO}\\[0.3in]

		
		\normalsize \textbf{MARCO MÉNDEZ ATIENZA}\\[0.3in]
		
		\small \textbf{PROMOCIÓN 2020-2022}\\[0.4in]
		
		\small \textbf{ASESORA:}\\[0.4in]
		
		\small \textbf{DRA. AURORA RAMÍREZ ÁLVAREZ}\\[0.4in]
		
		\small MAYO 2022

	\end{center}

}


\end{titlepage}


\begin{titlepage}

\vspace*{\fill}
	\begin{Huge}
		\textbf{Agradecimientos} \\[0.3in]
	\end{Huge}

\normalsize{\textbf{Quiero agradecer a...}}

Mi papá y mamá, cuyo apoyo y amor durante toda mi vida han sido determinantes de la persona que soy.

El Colegio de México, que me brindó los retos y herramientas para mejorar como persona y economista.

Mi asesora, Aurora Ramírez, por su guía y asesoría durante varios meses y sin las cuales este documento habría sido muy diferente.

Mis amigxs de toda la vida, lxs cuales no han dejado de escucharme y divertirme durante muchísimos años.

Mis amigxs y compañerxs de generación, y de otras generaciones, cuya ayuda y consejos durante dos años fueron esenciales para culminar exitosamente la Maestría. Gracias especiales a Job y sus habilidades de programación.

Todxs mis profesorxs del posgrado, y a sus enseñanzas académicas y personales.

\vspace*{\fill}
	
\end{titlepage}

\begin{titlepage}
\
\end{titlepage}


\begin{titlepage}
	
	\vspace*{\fill}
	\begin{Huge}
		\textbf{Resumen} \\[0.3in]
	\end{Huge}
	
	\normalsize{Los determinantes de la violencia han sido estudiados sustancialmente en diversas disciplinas y a partir de diversos enfoques. No obstante, el papel de las armas en ella, y más específicamente en el caso mexicano, es un tema sumamente fértil aún. Dada la complejidad de la violencia en nuestro país, en donde se suelen postular explicaciones o correlaciones a esta como la pobreza, la corrupción, el narcotráfico o el Estado de derecho, este trabajo identifica no solo un vínculo con una variable menos estudiada como las armas, sino su causalidad con el crimen.
	
	Así, este trabajo explota la diferenciación de un programa de canje de armas de fuego en la Ciudad de México, mucho más exitoso y sustantivo que el resto del país, como explicación de la disminución de los delitos con armas de fuego en la capital. A través del uso del Método de Control Sintético, se construye una unidad artificial de la Ciudad, con el objetivo de comparar la incidencia delictiva realmente observada con aquello que hubiera ocurrido de no haberse implementado el programa de canjes.
	
	Los resultados no solo evidencian que el programa contribuyó a la disminución de los delitos con armas de fuego, sino que son robustos a diversos cuestionamientos plausibles. En resumidas cuentas, los hallazgos apuntan a que el efecto de la intervención es sustantivo, solo afectó a los delitos estudiados y solo en la capital del país, a partir del periodo de inicio, y cuya magnitud ha sido creciente hasta los últimos datos reportados.
	}
	\vspace*{\fill}
	
\end{titlepage}

\begin{titlepage}
	\
\end{titlepage}


